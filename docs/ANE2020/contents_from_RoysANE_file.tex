%
% Daniele: this file is only temporary for Roy's disposal and understanding.
%

\section{Introduction}
\label{sec:intro}

The distribution of the neutron flux in the reactor core is described by the neutron transport equation. Transport calculations on a full core scale can be a highly intensive computational task. To overcome this difficulty, faster (but less accurate) multigroup neutron diffusion solvers are often used. Indeed, future Gen-IV reactor designs are characterized by strong heterogeneity in the core and modern calculation schemes evolve towards best-estimate codes, aiming at high accuracy for these new configurations. Hence, the accuracy of diffusion calculations is investigated. A crucial issue in obtaining an accurate diffusion calculation is the formulation of the diffusion coefficient. The calculation of this parameter should be based on physical insights from the full transport equation, such that the resulting (transport-corrected) diffusion approximation can capture the transport phenomena of interest.

% Daniele: I found this paragraph commented, but I think that the two following sections should be properly introduced in this section. Therefore, I am uncommenting it.
The main hypothesis of this research, called the Ronen Method (RM)~\cite{Ronen-2004,Tomatis-2011}, is based on iterative calculations of the multigroup diffusion coefficients, driven by accurate relations between the neutron current density and the neutron flux as derived from the integral transport equation. The iterative scheme calculates new flux distributions using a diffusion solver but with a (spatially) modified diffusion coefficient according to the current calculated using the integral transport equation. Alternatively, the Coarse Mesh Finite Differences (CMFD) scheme can be enforced in the discretized form of the diffusion equation to reproduce the transport current by means of a drift term. The implementation of one its variants, the partial current formulation of the CMFD (pCMFD), is studied in this work.
%
% The direct resolution of the integral relation between the diffusion coefficient and the transport current would imply the inversion of large matrices, with poor control of their conditioning. Nonetheless, the integral current can be enforced in the Coarse Mesh Finite Difference (CMFD) discretized form of the diffusion equation, which avoids the numerical issues resulting from small flux gradients. This is the option adopted in this study.

% ------------------------------------------
\section{Coarse Mesh Finite Differences}
\label{sec:CMFD-intro}

The CMFD has been largely used in nodal diffusion codes as a technique for storage reduction, allowing for separate resolution of many two nodes problems in an iterative scheme~\cite{Smith-1983,Lawrence-1986}. Recently, multi-dimensional transport codes have used it as acceleration technique and the convergence of different implementations was studied in simple homogeneous slab problems~\cite{Jarrett-2016,Shen-2019}.

The use of a coarse mesh for the discretization of the problem is generally causing considerable error by truncation applied on the finite difference approximations of the derivative terms. On the other hand, it yields very fast calculations. The CMFD method recovers corrective terms by exploiting functional relations between the (scalar) flux and the current density, which appear in the neutron balance equation. The balance equation is solved for the flux, so that local models seek more accurate estimates of the current in order to derive these corrections. In principle, the corrective terms can be computed with a different physics other than diffusion, which is always solved at the coarse level. This facilitates its application to accelerate complex multi-dimensional transport codes. This acceleration is actually implemented by rebalancing the scalar flux, with the optional use of some relaxation~\cite{Park-2017}. Hence, a non-linear iterative scheme arises between the CMFD solver and the computation of the local corrections, which use the latest estimates of the flux available from the diffusion solver.

Second-order SPN equations can be solved by ordinary diffusion codes~\cite{Larsen-1993}. Yamamoto et Al.~\cite{Yamamoto-2016} noticed numerical instabilities when correcting pseudo-currents obtained solely by the second-order moments of the expanded SP3 flux. This is because this quantity is generally smaller than the scalar flux (0-th moment) and can change its sign. They proposed to modify the balance equation for the second moment in order to fix this numerical behavior. The modification consisted of a shift with a positive quantity of the second moment so that their difference between neighboring quantities is unchanged, and their sum becomes positive and arbitrarily large.

When partial currents can be computed, the pCMFD method~\cite{Cho-2003} can be used to determine two distinct corrective terms per positive and negative partial currents, instead of having a single degree of freedom per cell surface. Another version of the CMFD called artificial diffusion~\cite{Zhu-2016,Jarrett-2016} redefines the diffusion coefficient as the sum of the original value and a complement derived from the current difference, which is still considered as proportional to the flux gradient. This scheme can potentially lead to negative diffusion constants.

The CMFD (and its variants) was also chosen for the implementation of the RM, where an integral transport operator estimated higher order currents to calculate the correction terms. Contrary to the previous implementations, no operator was inverted to get a new higher order approximation of the flux solution. As well, this offered a fast and more accurate estimate of the current too.


% ------------------------------------------
\section{The Ronen Method}
\label{sec:RM}

In 2004, Ronen~\cite{Ronen-2004} suggested to derive corrected diffusion coefficients using Fick's law and more accurate estimations of the neutron currents by means of integral transport operators, whereas the neutron flux was resolved by diffusion theory. Denoting the integral transport and diffusion currents by $\bm{J}\rE$ and $\bm{J}^D\rE$, respectively, the Ronen idea is based on adapting the diffusion coefficient such that
\begin{equation}
  \label{eq:Fick}
  \bm{J}^D\rE = -D\rE\bnabla\phi\rE \cong \bm{J}\rE.
\end{equation}

Since these accurate estimates of the currents were based on a known flux distribution, it was also suggested to execute new diffusion calculations, thus updating iteratively the diffusion coefficients in the global calculation according to
\begin{equation}
  \label{eq:RM-it}
  D^{(k+1)}\rE = - \frac{\lvert \bm{J}^{(k)}\rE \rvert}{\lvert \bnabla \phi^{(k)}\rE \rvert},
\end{equation}
where $k$ is the iteration index. The use of a tensor notation is needed for the diffusion coefficient in general multidimensional problems.

The motivation for this method was to overcome the inherent limitation of Fick's law requiring smooth flux gradients and thus weakly absorbing medium. Nevertheless, isotropic scattering remained as a basic postulate. For example, in a one-dimensional homogeneous slab with void boundary conditions, Eq.~\eqref{eq:RM-it} takes the following form~\cite{Ronen-2004}
\begin{equation}
  \label{eq:RM-it-1D-slab}
  D^{(k+1)}(x,E) = \left. -\frac{1}{2} \int_0^a dx'
    {E_2[\sigma(E) \lvert x-x' \rvert]\operatorname{sign}(x-x')q^{(k)}(x',E)}
    \middle/ \frac{\partial}{\partial x}\phi^{(k)}(x,E),\right.
\end{equation}
where the source term $q^{(k)}$ is calculated using the multigroup diffusion model with the available scalar flux, and the updated diffusion coefficient $D^{(k+1)}$ is evaluated using the integral expression for the transport current.

In 2011, Tomatis and Dall'Osso~\cite{Tomatis-2011} provided a numerical demonstration of the simple slab problem. They chose the CMFD scheme in the diffusion solver for taking into account the new currents estimated by the integral transport operator.
%
%They proposed to update non-linearly the diffusion coefficients at the cell interfaces of the spatial mesh by the coarse mesh finite differences method (CMFD).
%
This technique can avoid indeterminate divisions in case of vanishing flux gradients and reproduces the transport effects under a drift term. It was observed that the RM could drive the flux distribution closer to the transport reference solution. As expected, the largest errors were located near the boundary, where the transport effects are more pronounced, slowly decreasing even after many iterations.% Their work was limited to the homogeneous slab, and resolved neutron diffusion on the same discretized spatial mesh used for the reference from neutron transport.


% ------------------------------------------
\section{Numerical Implementation}
\label{sec:RM-num}

The neutron balance equation for a $k$-eigenvalue problem integrated in the $i$-th element of a one-dimensional mesh is (see Fig.~\ref{fig:mesh1D})
\begin{equation}
  \label{eq:balance}
  J_{i+\hzi,g} - J_{i-\hzi,g} + \sigma_{i,g} \phi_{i,g} = q_{i,g},
\end{equation}
where $J$ denotes the net current at the cell surface, $\sigma$ denotes the total cross section and the subscript $g$ denotes the energy group. $q$ is the isotropic neutron source defined as
\begin{equation}
  \label{eq:qsrc}
  q_{i,g} = \sum\limits_{g'=1}^G\sigma_{s,0,i,g\leftarrow g'}\phi_{i,g'} +
  \frac{\chi_g}{\keff}\sum\limits_{g'=1}^G\nu\sigma_{f,i,g'}\phi_{i,g'},
\end{equation}
where $\sigma_0$ is the isotropic scattering cross section, $\chi$ is the fission spectrum, $\keff$ is the effective multiplication factor (the eigenvalue), $\nu$ is the average number of neutrons emitted per thermal fission, and $\sigma_f$ is the fission cross section. As well, integer and rational subscripts indicate node-averaged and interface quantities, respectively.
%
\begin{figure}[htbp]
	\centering
	\begin{tikzpicture}
	\tikzstyle{every node}=[font=\footnotesize]
	\draw[thick] (0,0) -- (3,0);
	\node[anchor=south] at (3.5cm,-5pt) {$\ldots$};
	\draw[thick] (4,0) -- (7,0);
	\node[anchor=south] at (7.5cm,-5pt) {$\ldots$};
	\draw[thick,->] (8,0) -- (11,0) node[anchor=west] {$x$};
	
	\draw ( 0 cm,2pt) -- ( 0 cm,-2pt) node[anchor=north] {$-\hzi$};
	\node[anchor=south] at (1cm,1pt) {0};
	\draw ( 2 cm,2pt) -- ( 2 cm,-2pt) node[anchor=north] {$+\hzi$};
	
	\draw ( 4.5 cm,2pt) -- ( 4.5 cm,-2pt) node[anchor=north] {$i-\hzi$};
	\node[anchor=south] at (5.5cm,1pt) {$i$};
	\draw ( 6.5 cm,2pt) -- ( 6.5 cm,-2pt) node[anchor=north] {$i+\hzi$};
	
	\draw ( 8.5 cm,2pt) -- ( 8.5 cm,-2pt) node[anchor=north] {$I-\sfrac{3}{2}$};
	\node[anchor=south] at (9.5cm,1pt) {$I-1$};
	\draw ( 10.5 cm,2pt) -- ( 10.5 cm,-2pt) node[anchor=north] {$I-\hzi$};
	\end{tikzpicture}
	\caption{Notation of the one-dimensional mesh.}
	\label{fig:mesh1D}
\end{figure}

The net and partial currents in the case of isotropic scattering are calculated using the following expressions derived from the integral neutron transport equation~\cite{Tomatis-2011,Tomatis-2019,Gross-2020}
\begin{subequations}
  \label{eq:par-curr-calc}
  \begin{equation}
    \label{eq:par-curr-calc-plus}
    \begin{split}
      \jp_{i+\hzi,g} &=  \frac{1}{2}
        E_3[\mcL_g(x_{-\hzi},x_{i+\hzi})] \phi_{x_{-\hzi},g} \\
        & + \frac{1}{2}\sum\limits_{j=0}^i \frac{q_{j,g}}{\sigma_{j,g}}
            \left\{ E_3\left[ \mcL_g(x_{j+\hzi},x_{i+\hzi}) \right]
                  - E_3\left[ \mcL_g(x_{j-\hzi},x_{i+\hzi}) \right]
            \right\}
    \end{split}
  \end{equation}
  %
  \begin{equation}
    \begin{split}
      \label{eq:par-curr-calc-minus}
      \jm_{i+\hzi,g} &=  \frac{1}{2}
        E_3[\mcL_g(x_{i+\hzi},x_{I-\hzi})] \phi_{x_{I-\hzi},g} \\
        & + \frac{1}{2}\sum\limits_{j=i+1}^{I-1} \frac{q_{j,g}}{\sigma_{j,g}}
            \left\{ E_3\left[ \mcL_g(x_{i+\hzi},x_{j-\hzi}) \right]
                  - E_3\left[ \mcL_g(x_{i+\hzi},x_{j+\hzi}) \right]
            \right\},
    \end{split}
  \end{equation}
where $\mcL_g(x',x)\equiv \int_{x'}^{x} \sigma_g(x'')dx''$ is the optical length and the net current is $J _{i+\hzi,g} = \jp _{i+\hzi,g} - \jm _{i+\hzi,g}$, $\forall i$. Only the zeroth moment of the angular flux, that is the scalar flux after an expansion in angle on the spherical harmonics, is used at the boundary to include the contribution of possible incoming currents. \eqs{eq:par-curr-calc} are expected to yield more accurate estimates of the currents even when using a flux distribution obtained from diffusion theory. This is the main assumption of the Ronen Method.
\end{subequations}
% \daniele{In case additional space will be needed, it will be possible to write \eqs{eq:par-curr-calc} in a condensed form by using $\pm$.}

% ------------------------------------------
\subsection{CMFD Implementation}
\label{sec:RM-CMFD}

In classic CMFD implementation, the term $\delta J_{i+\hzi,g}$ is added to the diffusion current $\jD_{i+\hzi,g}$ in order to reproduce the net current $J_{i+\hzi,g}$. Currents are only computed on the surface elements of a coarse mesh in space to save memory and runtime \cite{Smith-1983}. Besides, the net current can be determined from some other expression which is physically or numerically more accurate. Therefore, this new term acts as a correction for the diffusion current. The diffusion current is generally approximated by discretizing Fick's law, and by finite differences, according to
\begin{equation}
  \label{eq:JD}
  \jD_{i+\hzi,g} \cong - D_{i+\hzi,g}
    \frac{\phi_{i+1,g} - \phi_{i,g}}{(\Delta_{i+1} + \Delta_i)/2},
\end{equation}
where the diffusion coefficient at the interface is volume-averaged as
\begin{equation}
  \label{eq:Ds}
  D_{i+\hzi,g} = \frac{\Delta_i D_{i,g} + \Delta_{i+1} D_{i+1,g}}
{\Delta_i + \Delta_{i+1}}.
\end{equation}
%
About the RM, more accurate estimates of the net current come from the integral expressions of \eqs{eq:par-curr-calc}, still using the scalar flux in the emission source $q_{i,g}$. In linearly anisotropic problems, the net current can also be considered to compute the source. Once the diffusion currents and the net currents by integral transport are calculated using the most recent flux values, the correction terms can be obtained on the cell interfaces. The discretized form of the terms $\delta J_{i+\hzi,g}$ reproduce a centered drift-advection term to avoid possible indeterminate division by zeros in case of flat flux~\cite{Smith-1983,Tomatis-2011}:
\begin{equation}
  \label{eq:dJ}
  \delta J_{i+\hzi,g} = -\delta D_{i+\hzi,g}
    \frac{\phi_{i+1,g} + \phi_{i,g}}{(\Delta_{i+1} + \Delta_i)/2} =
  J_{i+\hzi,g} - \jD_{i+\hzi,g},
\end{equation}
for $i = 0, \ldots, I-2$. The division by the spatial differences can be removed because the definition of $\delta D$ is arbitrary.

The CMFD, eigenvalue balance equation are 
\begin{equation}
  \label{eq:CMFD-diff}
  J^+_{D,i+\hzi,g} + \delj^+_{i+\hzi,g} - J^-_{D,i-\hzi,g} - \delj^-_{i-\hzi,g} + \sigma_{i,g}\phi_{i,g}=
  \frac{\chi_g}{\keff}\sum\limits_{g'=1}^G\nu\sigma_{f,i,g'}\phi_{i,g'} + \sum\limits_{g'=1}^G\sigma_{s,i,g\leftarrow g'}\phi_{i,g'}. 
\end{equation}
Using Eqs.~\eqref{eq:JD}, \eqref{eq:dJ}, and rearranging, yields
\begin{equation}
  \label{eq:CMFD-diff-rearranged}
  \begin{split}
  \left( \frac{-D_{i-\hzi,g} + \dd_{i-\hzi,g}}{\dx_{i-1} + \dx_i}\right)\phi_{i-1,g}
  & + \left(\frac{D_{i+\hzi,g} - \dd_{i+\hzi,g}}{\dx_{i+1} + \dx_i} + \frac{D_{i-\hzi,g} + \dd_{i-\hzi,g}}{\dx_i + \dx_{i-1}} + \frac{\sigma_{i,g}}{2}\right) \phi_{i,g} \\
  & + \left( \frac{-D_{i+\hzi,g} - \dd_{i+\hzi,g}}{\dx_{i+1} + \dx_i} \right) \phi_{i+1,g} = \frac{1}{2}q_{i,g}.
  \end{split}
\end{equation}

Boundary conditions applies on the diffusion current as usual by means of an extrapolated length $\zeta$, $\zeta \jD \cong D \phi$. Mark or Marshak approximations, or other constants derived from transport theory can be used for $\zeta$ \cite{meghreblian1960reactor}. About the correction terms instead, it is simply $\delta J = -\delta D \phi$ with the boundary flux approximated at first order by the cell-averaged value.

% ------------------------------------------
\subsection{pCMFD Implementation}
\label{sec:RM-pCMFD}

The CMFD introduces in the numerical scheme a free parameter per cell surface through the correction term to adjust the diffusive solution locally. Since separate calculations of the partial currents are possible, two free parameters could be used in the discretized form of the diffusion equation. This idea leads to the development of the pCMFD scheme \cite{cho2003comparison}.

Two corrective terms are then introduced: $J = \jp - \jm = \jD + \left(\delta J^+ - \delta J^-\right)$. An upwind definition with respect to the direction of flight of neutrons is used for the corrections~\cite{Jarrett-2016,Zhu-2016}:
\begin{equation}
\label{eq:CCF-sub}
\delta \jp _{i+\hzi} = \delta D^+ _{i+\hzi} \phi_i
\quad \text{and} \quad
\delta \jm _{i+\hzi} = -\delta D^- _{i+\hzi} \phi_{i+1}.
\end{equation}
The current by diffusion is distributed on the partial currents in equal parts in order to compute the new corrective coefficients $\delta D^{\pm}$,
\begin{subequations}
	\label{eq:delta-D-pCMFD}
	\begin{align}
	J^+_{i+\hzi} &= \frac{1}{2}\jD_{i+\hzi} + \delta D^+_{i+\hzi} \phi_i, \label{eq:delta-D-pCMFDp}\\
	J^-_{i+\hzi} &= - \frac{1}{2}\jD_{i+\hzi} + \delta D^-_{i+\hzi} \phi_{i+1}, \label{eq:delta-D-pCMFDm}
	\end{align}
\end{subequations}
still for $i=0, \ldots, I-2$. The diffusion current is not halved at the boundary, using only \eq{eq:delta-D-pCMFDm} or \eq{eq:delta-D-pCMFDp} at the left and at the right boundaries, respectively. Finally, the implementation of CMFD can be upgraded to support the pCMFD scheme with minimal changes.
% -----------------------------------------------------------------------------
%
% Daniele: the following proposition has shown to be inconsistent.
%
% Thanks to P$_1$ diffusion theory, the partial currents at any point $x$ of the slab are approximated as
% \begin{equation}
% \label{eq:partial-current}
% \jpm(x) \cong \frac{\phi(x)}{4} \pm \frac{J (x)}{2},
% \end{equation}
% still verifying the expression for the net current $J = J^+ - J^-$. \eq{eq:partial-current} suggests introducing two different correction terms at each cell interface depending on the sign of the partial currents
% \begin{equation}
%   \label{eq:PCCF}
%   \jpm _{i+\hzi} = \frac{1}{4} \phi_{i+\hzi} \pm \frac{1}{2}
%     \left(\jD_{i+\hzi} + \delta \jpm_{i+\hzi}\right),
% \end{equation}
% and they must fulfill again the net current $J = \jD + (\delta \jp + \delta \jm)/2$. An upwind definition with the respect to the direction of flight of neutrons is~\cite{Jarrett-2016,Zhu-2016}
% \begin{equation}
% \label{eq:CCF-sub}
% \delta \jp _{i+\hzi} = -2\delta D^+ _{i+\hzi} \phi_i
% \quad \text{and} \quad
% \delta \jm _{i+\hzi} = -2\delta D^- _{i+\hzi} \phi_{i+1}.
% \end{equation}
% %
% Solving for the partial current correction factors gives
% \begin{equation}
%   \label{eq:ccfsss1}
%   \delta D^+ _{i+\hzi} = \frac{\frac{1}{4}\phi_{i+\hzi}
%   	+ \frac{1}{2}\jD _{i+\hzi} - \jp _{i+\hzi}}{\phi_i}
%   \quad \text{and} \quad
%   \delta D^- _{i+\hzi} = \frac{-\frac{1}{4}\phi_{i+\hzi}
%   	+ \frac{1}{2}\jD _{i+\hzi} + \jm _{i+\hzi}}{\phi_{i+1}},
%   \end{equation}
% where the flux at the interface can be simply volume-averaged too, see~\eq{eq:Ds} for instance.
% -----------------------------------------------------------------------------
The descretized form of the diffusion equations, with pCMFD implementations, is
\begin{equation}
  \label{eq:pCMFD-diff1}
  J_{D,i+\hzi,g}^+ +\left( \delj_{i+\hzi,g}^+ - \delj_{i+\hzi,g}^- \right) - J_{D,i-\hzi,g}^+ + \left( + \delj_{i-\hzi,g}^+ - 
  \delj_{i-\hzi,g}^- \right) + \sigma_{i,g}\phi_{i,g} = q_{i,g},
\end{equation}
where
\begin{equation}
  \label{eq:q}
  q_{i,g} = \frac{\chi_g}{\keff}\sum\limits^G_{g'=1}\nu\sigma_{f,g'}\phi_{i,g'} + \sum\limits_{g'=1}^G \sigma_{s,i,0,g\leftarrow g'}\phi_{i,g'}.
\end{equation}
By using Eqs.~\eqref{eq:CCF-sub}--\eqref{eq:q}, one gets
\begin{equation}
  \label{eq:pCMFD-diff2}
  \begin{split}
  -2D_{i+\hzi}\frac{\phi_{i+1,g} - \phi_{i,g}}{\dx_{i+1}+\dx_{i}} +\left( \dd^+_{i+\hzi,g}\phi_{i,g} \right.&\left. - \dd^-_{i+\hzi,g }\phi_{i+1,g}\right)
  +2D_{i-\hzi}\frac{\phi_{i,g} - \phi_{i-1,g}}{\dx_{i}+\dx_{i-1}} \\ 
  &- \left( \dd^+_{i-\hzi,g}\phi_{i-1,g} -\dd^-_{i-\hzi,g}\phi_{i,g}\right) +\sigma_{i,g}\phi_{i,g} = q_{i,g}.
  \end{split}
\end{equation}
Setting $\tild_{i+\hzi,g} = \frac{2D_{i+\hzi,g}}{\dx_{i+1}+\dx_{i}}$, and rearranging the above term, 
\begin{equation}
  \label{eq:pCMFD-rearrange}
  \begin{split}
  \left(-\tild_{i-\hzi,g}-\dd^+_{i-\hzi,g}\right)\phi_{i-\hzi,g} + &\left( \tild_{i+\hzi,g} + \dd^+_{i+\hzi,g}+ \tild_{i-\hzi,g} + \sigma_{i,g}\right)\phi_{i,g} \\
  &+ \left( -\tild_{i+\hzi,g} - \dd^-_{i+\hzi,g}\right)\phi_{i+1,g} = q_{i,g}.
  \end{split}
\end{equation}

In a weakly absorbing medium and several mean free paths away from the boundary or a strong absorber, all correction factors $\delta D^\pm$ are expected to vanish and the net current must be fully reproduced by the diffusion current. Of course, this must be verified by the classic CMFD scheme too.

% ------------------------------------------
\section{Results}
\label{sec:res}

xy

% ------------------------------------------
\section{Conclusions}
\label{sec:conc}

Present your summary and conclusions here.
