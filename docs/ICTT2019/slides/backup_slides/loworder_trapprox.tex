\begin{frame}
  \frametitle{Approaches to improve diffusion}

  \begin{block}{\bf Second-order transport approximations}
    {\vspace{3mm}
    \begin{itemize} \Large
      \setlength{\itemsep}{5mm}
      \item Different ``diffusive-like'' approximations of transport exist
      \item They can be implemented in existing diffusion-codes
    \end{itemize}
    }
  \end{block}
  \vfill
  \begin{exampleblock}{\bf \ldots{a} few examples:}
  {\vspace{3mm}
  \begin{itemize} \large
    \setlength{\itemsep}{5mm}
    \item \textcolor{ceablue1}{Diffusion tensors} to treat material heterogeneities with a \emph{drift component} \cite{boffi1972tensorial}
    \item Simplified $P_N$ or \textcolor{ceablue1}{$SP_N$} \cite{gelbard1960application, brantley2000simplified}, the \textcolor{ceablue1}{$A_N$} formulation \cite{PieroSNandAN, coppa2016an}
    \item \textcolor{ceablue1}{Quasi-diffusion} with the Eddington factor \cite{pounders2009diffusion}
    \item more accurate evaluation of the current in the actual system, i.e. the \textcolor{ceablue1}{Ronen Method} \cite{ronen2004accurate}
  \end{itemize}
  }
  \end{exampleblock}
\end{frame}
