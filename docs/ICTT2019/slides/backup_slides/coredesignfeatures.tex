\begin{frame}
  \frametitle{Regularity, Periodicity and Modularity}
  \begin{columns}
  \vspace{-5mm}
    \column{0.425\linewidth}
    \begin{block}{}{ \sl
    ``A rich range of reactor types has been developed in years to accomplish various operating objectives and to overcome technological limitations, thus using different coolants and nuclear fuels''.}
    \end{block}
    %\vspace{5mm}
    %
    \begin{exampleblock}{}
    \begin{flushright}
      \textsc{LWR (PWR \& BWR), HWR, Gas-cooled and Graphite-moderated, LMFBR / SFR}, (+ \emph{research reactors}).
    \end{flushright}
    \end{exampleblock}
    \vspace{5mm}
    %
    \fbox{\parbox{\textwidth}{ \centering
    Homogeneous~vs.~\emph{Heterogeneous systems}\\
    $\implies$ higher resonance escape probability.
    }}\vspace{5mm}
    %
    \begin{block}{\sl Features of the active core geometry}
      \textcolor{ceared1}{lattice} (\emph{regularity}) and \textcolor{ceared1}{assemblies} (\emph{modularity}), \emph{periodically} filling the multiplying system to achieve the target power.
    \end{block}
    %
    \column{0.575\linewidth} \centering
    {
    Hierarchical structure of a water reactor (PWR top, BWR bottom).}\\
    \includegraphics[width=0.8\textwidth]{figures/HierarcicalStructurePWR.png}\\[2mm]
    \includegraphics[width=0.8\textwidth]{figures/bwr_bundle.jpeg}
  \end{columns}
\end{frame}
